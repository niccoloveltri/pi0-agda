% This is samplepaper.tex, a sample chapter demonstrating the
% LLNCS macro package for Springer Computer Science proceedings;
% Version 2.20 of 2017/10/04
%
\documentclass[runningheads]{llncs}
%
\usepackage{graphicx}
% Used for displaying a sample figure. If possible, figure files should
% be included in EPS format.
%
% If you use the hyperref package, please uncomment the following line
% to display URLs in blue roman font according to Springer's eBook style:
% \renewcommand\UrlFont{\color{blue}\rmfamily}

\usepackage{proof}
\usepackage{amsmath}
%\usepackage{amsthm}
\usepackage{amssymb}
\usepackage{mathtools}

\newcommand{\N}{\mathbb{N}}
\newcommand{\Pio}{$\Pi^{\mathsf{o}}$} 
\newcommand{\Set}{\mathsf{Set}}
\newcommand{\comp}{\circ}

% Syntax Pi0
\newcommand{\Ty}{\mathsf{Ty}}
\newcommand{\Var}{\mathsf{Var}}
\newcommand{\id}{\mathsf{id}}
\newcommand{\lr}{\longleftrightarrow}
\newcommand{\fold}{\mathsf{fold}}
\newcommand{\unfold}{\mathsf{unfold}}
\newcommand{\sub}{\mathsf{sub}}
\newcommand{\trace}{\mathsf{trace}}

% Delay macros
\newcommand{\Delay}{\mathsf{Delay}\,}
\newcommand{\now}{\mathsf{now}}
\newcommand{\later}{\mathsf{later}}
\newcommand{\never}{\mathsf{never}}
\newcommand{\dn}{\downarrow}
\newcommand{\bind}{\mathsf{bind}}
\newcommand{\str}{\mathsf{str}}
\newcommand{\D}{\mathbb{D}}
\newcommand{\Dapprox}{\mathbb{D}_{\approx}}

\begin{document}
%
\title{Type-theoretic Semantics of \Pio\ via the Delay Monad \thanks{Supported by organization x.}}
%
%\titlerunning{Abbreviated paper title}
% If the paper title is too long for the running head, you can set
% an abbreviated paper title here
%
\author{Robin Kaarsgaard \inst{1}\orcidID{0000-0002-7672-799X} \and
Niccol\`o Veltri\inst{2}\orcidID{0000-0002-7230-3436}}
%
\authorrunning{Kaarsgaard and Veltri}
% First names are abbreviated in the running head.
% If there are more than two authors, 'et al.' is used.
%
\institute{DIKU, Department of Computer Science, University of
    Copenhagen\\ \email{robin@di.ku.dk} \and
Department of Computer Science, IT University of
    Copenhagen\\ \email{nive@itu.dk}}
%
\maketitle              % typeset the header of the contribution
%
\begin{abstract}
The abstract should briefly summarize the contents of the paper in
150--250 words.

\keywords{First keyword  \and Second keyword \and Another keyword.}
\end{abstract}

\section{Introduction}\label{sec:intro}

\section{Syntax of \Pio}\label{sec:syntax}

\subsection{Types}

\begin{figure}
\[
\arraycolsep=8pt\def\arraystretch{2.8}
\begin{array}{|ccc|}
\hline 
\infer{0 : \Ty_n}{} 
& \infer{1 : \Ty_n}{}
& \infer{\Var\,i : \Ty_n}{i \in [0..n)}  \\
\infer{A + B : \Ty_n}{A : \Ty_n & B : \Ty_n}
& \infer{A \times B : \Ty_n}{A : \Ty_n & B : \Ty_n}
& \infer{\mu\,A : \Ty_n}{A : \Ty_{n+1}} \\
\hline
\end{array}
\]
\caption{Types of \Pio.}
\label{fig:types}
\end{figure}

\subsection{Programs}

\begin{figure}
\[
\arraycolsep=8pt\def\arraystretch{2.8}
\begin{array}{|cc|}
\hline
\infer{\id : A \lr A}{}
& \infer{g \cdot f : A \lr C}{g : B \lr C & f : A \lr B} \\
\infer{f + g : A + B \lr C + D}{f : A \lr C & g : B \lr D} 
& \infer{f \times g : A \times B \lr C \times D}{f : A \lr C & g : B \lr D} \\
\infer{\lambda_{+} : 0 + A \lr A}{}
& \infer{\lambda_{+}^{-1} : A \lr 0 + A}{} \\
\infer{\lambda_{\times} : 1 \times A \lr A}{}
& \infer{\lambda_{\times}^{-1} : A \lr 1 \times A}{} \\
\infer{\alpha_{+} : (A + B) + C \lr A + (B + C)}{}
& \infer{\alpha_{+}^{-1} : A + (B + C) \lr (A + B) + C}{} \\
\infer{\alpha_{\times} : (A \times B) \times C \lr A \times (B \times C)}{}
& \infer{\alpha_{\times}^{-1} : A \times (B \times C) \lr (A \times B) \times C}{} \\
\infer{\sigma_{+} : A + B \lr B + A}{}
& \infer{\sigma_{\times} : A \times B \lr B \times A}{} \\
\infer{\kappa : 0 \times A \lr 0}{} 
& \infer{\delta : (A + B) \times C \lr (A \times C) + (B \times C)}{} \\
\infer{\kappa^{-1} : 0 \lr 0 \times A}{}
& \infer{\delta^{-1} : (A \times C) + (B \times C)  \lr (A + B) \times C}{} \\
\infer{\fold : \sub \, A \, (\mu \,A) \lr \mu \,A}{} 
& \infer{\unfold : \mu\,A \lr \sub \, A \, (\mu \,A)}{} \\
\multicolumn{2}{|c|}{\infer{\trace \,f: B \lr C}{f : A + B \lr A + C}} \\
\hline
\end{array}
\]
\caption{Programs of \Pio}
\label{fig:programs}
\end{figure}

\section{Delay Monad}\label{sec:delay}

The coinductive delay datatype was first introduce by Capretta for
representing general recursive functions in Martin-L\"of type theory
\cite{Capretta05}.  Given a type $A$, elements of $\Delay A$ are
possibly non-terminating ``computations'' returning a value of $A$
whenever they terminate. Formally, $\Delay A$ is defined as a
coinductive type with the following introduction rules:
\[
\infer={\now\,a : \Delay A}{x : A}
\quad
\infer={\later\,x : \Delay A}{x : \Delay A}
\]
(Here and in the following, single rule lines refer to an inductive
definition, double rule lines to a coinductive definition.) The
constructor $\now$ embeds $A$ into $\Delay A$, so $\now\,a$ represents
the terminating computation returning the value $a$. The constructor
$\later$ adds an additional unit of time delay to a computation. The
non-terminating computation $\never$ is corecursively defined as
$\never = \later \,\never$.

The delay datatype is a monad. The unit is the constructor $\now$,
while the Kleisli extension $\bind$ is corecursively defined as follows:
\begin{align*}
& \bind : (A \to \Delay B) \to \Delay A \to \Delay B \\
& \bind \,f \, (\now\,a) = f\,a \\
& \bind\,f\,(\later\,x) = \later\,(\bind\,f\,x)
\end{align*}
The delay monad, like any other monad on $\Set$, has a unique strength
operation which we denote by $\str : A \times \Delay B \to \Delay (A
\times B)$. Moreover, the delay datatype is a commutative monad.

The Kleisli category of the delay monad, that we call $\D$, has types
as objects and functions $f : A \to \Delay B$ as morphisms between $A$
and $B$. In $\D$, the identity map on an object $A$ is the constructor
$\now$, while the composition of morphisms $f : A \to \Delay B$ and $g
: B \to \Delay C$ is given by $f \diamond g = \bind\,f \comp g$.

The delay datatype allows us to program with partial functions, but
the introduced notion of partiality is intensional, in the sense that
computations terminating with the same value in a different number of
steps are considered different. To obtain an extensional notion of
partiality, which in particular allows the specification of a
well-behaved trace operator, we need to quotient the delay datatype by
weak bisimilarity.

Weak bisimilarity is defined in terms of convergence. A computation
$x : \Delay A$ converges to $a : A$ if it terminates in a finite
number of steps returning the value $a$. When this happens, we write
$x \dn a$. The relation $\dn$ is inductively defined by the rules:
\[
\infer{\now\,a \dn a}{}
\quad
\infer{\later\,x \dn a}{x \dn a}
\]
Two computations in $\Delay A$ are weakly bisimilar if they differ by
a finite number of applications of the constructor $\later$. This
informal statement can be formalized in several different but
logically equivalent ways \cite{ChapmanUV19}. For this work we chose the
following definition:
\[
x \approx y = (a : A) \to (x \dn a \to y \dn a) \times (y \dn a \to x \dn a) 
\]
The definition says that $x$ and $y$ are weakly bisimilar if, whenever
$x$ terminates, then $y$ also terminates with the same value, and vice
versa. Weak bisimilarity is an equivalence relation and it is a
congrence with respect to the $\later$ operation.  We call $\Dapprox$
the Kleisli category of $\Delay$ with homsets quotiented by pointwise
weak bisimilarity. This means that in $\Dapprox$ two morphisms $f$ and
$g$ are considered equal whenever $f \, a \approx g \, a$, for all
inputs $a$. When this is the case, we also write $f \approx g$.  In
this work we do not formally construct the quotients
$(A \to \Delay B)/{\approx}$, since we work in a type theory without
quotient types. Instead we require that every construction we perform
on morphisms in $\Dapprox$ respects weak bisimilarity. Technically, we
construct $\Dapprox$ as a category enriched in setoids CITE, where in
particular all the reasoning on morphisms is performed up-to weak
bisimilarity.

In previous work \cite{ChapmanUV19} we showed that, when working in a
type theory with quotient types a la Hofmann CITE, quotienting the
delay monad by weak bisimilarity has problematic consequences. In
fact, it does not seem possible to recover the monad structure on the
quotiented datatype $\mathsf{DelayQuot}\,A = \Delay A/{\approx}$
without assuming additional controversial principles
such as the axiom of countable choice. In particolar, countable choice
would be necessary for modelling the trace operator of \Pio. In
loc.cit. we showed that a possible way to avoid these issues is
forming the quotients of homsets, as in $(A \to \Delay B)/{\approx}$, instead of
quotienting the delay datatype directly. We would still not be able to form




\section{Elgot Iteration and Trace}\label{sec:elgot}

\section{Interpretation}\label{sec:interpretation}

\section{Conclusions}


%
%
%
%
% ---- Bibliography ----
%
% BibTeX users should specify bibliography style 'splncs04'.
% References will then be sorted and formatted in the correct style.
%
\bibliographystyle{splncs04}
\bibliography{paper}

\end{document}
