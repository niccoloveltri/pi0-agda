\documentclass[12pt,t]{beamer}

%% \mode<presentation>
%% {
%% %  \usetheme{Goettingen}
%%   \useoutertheme{default}
%%   \useinnertheme{rounded}
%%   \setbeamercovered{transparent}
%%   \usecolortheme{rose}
%% }


\usepackage{proof}

\usepackage{newlfont}

\usepackage{amssymb}
\usepackage{latexsym}
\usepackage{stmaryrd}
\usepackage{tikzit}
\input{figures/stringdiagrams.tikzstyles}

%\usepackage[all]{xy}

\usepackage{graphicx}
\usepackage{color}

\setbeamerfont*{itemize/enumerate body}{size=\normalsize}
\setbeamerfont*{itemize/enumerate subbody}{parent=itemize/enumerate body}
\setbeamerfont*{itemize/enumerate subsubbody}{parent=itemize/enumerate body}


%% \usepackage{stmaryrd}

%\usepackage{pgfpages}
%\pgfpagesuselayout{4 on 1}[a4paper,border shrink=5mm,landscape]


\newcommand{\red}[1]{\textcolor{red}{#1}}
\newcommand{\blue}[1]{\textcolor{blue}{#1}}


\newcommand{\N}{\mathbb{N}}
\newcommand{\C}{\mathbb{C}}
\newcommand{\Pio}{\ensuremath{\mathsf{\Pi}^{\mathsf{o}}}} 
\newcommand{\Set}{\mathsf{Set}}
\newcommand{\Par}{\mathsf{Par}}
\newcommand{\comp}{\circ}
% \newcommand{\U}{\mathcal{U}}
\newcommand{\copair}[2]{[#1,#2]}
\newcommand{\pair}[2]{\left< #1,#2 \right>}
\newcommand{\inl}{\mathsf{inl}}
\newcommand{\inr}{\mathsf{inr}}
\newcommand{\fst}{\mathsf{fst}}
\newcommand{\snd}{\mathsf{snd}}
\newcommand{\T}{\mathsf{T}}
\newcommand{\Inv}{\mathsf{Inv}}
\newcommand{\iter}{\mathsf{iter}}
\newcommand{\mapL}[1]{#1_{\mathsf{L}}}
\newcommand{\mapR}[1]{#1_{\mathsf{R}}}

% Syntax Pi0
\newcommand{\Ty}{\mathsf{Ty}}
\newcommand{\Var}{\mathsf{Var}}
\newcommand{\id}{\mathsf{id}}
\newcommand{\lr}{\longleftrightarrow}
\newcommand{\fold}{\mathsf{fold}}
\newcommand{\unfold}{\mathsf{unfold}}
\newcommand{\sub}{\mathsf{sub}}
\newcommand{\trace}{\ensuremath{\mathsf{trace}}}
\newcommand{\Z}{\mathsf{Z}}
\newcommand{\I}{\mathsf{I}}
\newcommand{\LR}{\iff}
\renewcommand{\dagger}{\mathsf{dagger}}

% Delay macros
\newcommand{\Delay}{\ensuremath{\mathsf{Delay}\,}}
\newcommand{\Maybe}{\mathsf{Maybe}}
\newcommand{\now}{\mathsf{now}}
\newcommand{\later}{\mathsf{later}}
\newcommand{\laterR}{\mathsf{laterR}}
\newcommand{\laterL}{\mathsf{laterL}}
\newcommand{\never}{\mathsf{never}}
\newcommand{\dn}{\downarrow}
\newcommand{\bind}{\mathsf{bind}}
\newcommand{\str}{\mathsf{str}}
\newcommand{\costr}{\mathsf{costr}}
\newcommand{\D}{\mathbb{D}}
\newcommand{\Dapprox}{\mathbb{D}_{\approx}}
\newcommand{\copairD}[2]{[#1,#2]_{\D}}
\newcommand{\pairD}[2]{\left< #1,#2 \right>_{\D}}
\newcommand{\inlD}{\mathsf{inl}_{\D}}
\newcommand{\inrD}{\mathsf{inr}_{\D}}
\newcommand{\fstD}{\mathsf{fst}_{\D}}
\newcommand{\sndD}{\mathsf{snd}_{\D}}
\newcommand{\piso}{\mathsf{isPartialIso}}
\newcommand{\pisoalt}{\mathsf{isPartialIsoCat}}
\newcommand{\iterD}{\mathsf{iter}_\D}
\newcommand{\iterDapprox}{\mathsf{iter}_{\Dapprox}}
\newcommand{\unfolding}{\mathsf{fixpoint}_\D}
\newcommand{\refl}{\mathsf{refl}}
\newcommand{\traceD}{\mathsf{trace}_\D}
\newcommand{\traceHD}{\mathsf{traceH}_\D}
\newcommand{\daggerD}{\mathsf{dagger}_\D}
\newcommand{\Orb}[3]{\mathsf{Orb}\,#1\,#2\,#3}
\newcommand{\done}{\mathsf{done}}
\newcommand{\mynext}{\mathsf{next}}
\newcommand{\reverseOrbit}{\mathsf{reverseOrb}}

% Semantics
\newcommand{\semTy}[1]{{\llbracket #1 \rrbracket}_\mathsf{Ty}}
\newcommand{\semMu}[1]{{\llbracket #1 \rrbracket}_\mu}
\newcommand{\semTm}[1]{{\llbracket #1 \rrbracket}_{\lr}}
\newcommand{\semEq}[1]{{\llbracket #1 \rrbracket}_{\!\!\LR}}
\newcommand{\semTrace}{\mathsf{trace}_{\D\simeq}}

\newcommand{\todo}[1]{\hl{TODO: #1}}
\newcommand{\todocite}[1]{\hl{[TODO: #1]}}

% =======================================================================


\begin{document}

% frontmatter

\title{Unguarded Iteration for Reversible Computation in the
  Delay Monad}


\author{Robin Kaarsgaard \inst{1} \and \underline{Niccol\`o Veltri} \inst{2}}
\institute{\inst{1} University of Copenhagen \and \inst{2} Tallinn University of Technology}
\date{Estonian-Latvian Theory Days'19, P\"arnu, 11 October 2019}

\begin{frame}

\maketitle

\end{frame}


\begin{frame}

  \frametitle{Reversible computation}

  \begin{itemize}

  \item Reversible computation: programs are not just deterministic when
    executed the forward direction, but also in the
    backward direction.
  \item It forms the basis for quantum computing.
  \item Applications in different areas: low-power computing, robotics,
    discrete event simulation, etc.

    \pause
    \vspace{\fill}
    
  \item \Pio, a reversible combinator calculus [James and Sabry'12].
  \item Simple yet expressive: semantics of typed reversible
    functional programming language Theseus.
    % Turing-complete reversible programming language
  \end{itemize}

\end{frame}

\begin{frame}

  \frametitle{\Pio: language overview [James and Sabry'12]}

  \begin{itemize}

  \item Types:
    \[
    A ::= \Z \, | \,A \oplus A \, | \,\I \,| \,A \otimes A \,| \,X \,|
    \,\mu X.A
    \]
  \item Programs $t : A \lr B$, reversible computations between types $A$
    and $B$.
  \item Crucially among those:
    \begin{itemize}          
    \item Inversion operation:
      \[
      \infer{\dagger \,f : B \lr A}{f : A \lr B}
      \]
    \item Trace combinator for iteration:
      \[
      \infer{\trace \,f: A \lr B}{f : A \oplus C \lr B \oplus C}
      \]
      % For iteration, 
    \end{itemize}    
  \end{itemize}
  
\end{frame}


\begin{frame}

  \frametitle{$\Pi$: a finitary subcalculus [Carette and Sabry'16]}

  \begin{itemize}

  \item $\Pi$, the finitary fragment of $\Pio$ : no recursive types,
    no trace combinator.

  \item $\Pi$ includes higher reversible computations: program
    equivalences $e : t \LR u$
    \vspace{\fill}
  \item (Formally, program equivalences correspond to the coherence laws of
    symmetric rig groupoids [Laplaza'72]).
%%  \item E.g., there is a term $\alpha_{\otimes} : A \otimes (B \otimes
%%    C) \lr (A \otimes B) \otimes C$ with a term equivalence
%%    corresponding to MacLane's pentagon:
%%    \[
%%    \mathsf{pentagon} : \alpha_{\otimes} \circ \alpha_{\otimes} \LR (\id \otimes
%%    \alpha_{\otimes}) \circ \alpha_{\otimes} \circ (\alpha_{\otimes}
%%    \otimes \id)
%%    \]
    
  \end{itemize}
  
\end{frame}

\begin{frame}
  \frametitle{$\Pi$ in Agda [Carette and Sabry'16]}

  \begin{itemize}

  \item Agda formalization of syntax and
    denotational
    semantics of $\Pi$.
  \item Types are interpreted in $\Set$, the universe of small types:
    \[
    \semTy A : \Set
    \]
  \item Programs are interpreted as type isomorphisms:
    \[
    \semTm {t : A \lr B} : \semTy A \simeq \semTy B
    \]
  \item Program equivalences are interpreted as equivalences of
    type isomorphisms.
    % Existence of homotopies between the functions underlying the
    % equivalences of types
  \end{itemize}
  
  
\end{frame}

\begin{frame}
  
  \frametitle{In this work: \Pio\ in Agda}

  \begin{itemize}
  \item Extending Carette and Sabry's work: Agda formalization of
    syntax and denotational semantics of the full \Pio\ calculus.
    \vspace{\fill}
  \item  Agda formalization at 
    \begin{center}
      \url{https://github.com/niccoloveltri/pi0-agda}
    \end{center}    
  \item Paper presented last Wednesday at MPC'19 in Porto (part of
    FM'19).

    
%%  \item Type theory is a total language, no way of specifying a trace
%%    operator on types.
%%  \item Carette and Sabry's semantic domain is not good for our
%%    purpouse.
%%    \pause
%%    \vspace{\fill}
%%  \item Need a framework supporting possibly non-terminating
%%    computations: use Capretta's coinductive delay monad.
%%    \\
%%    (The maybe monad is not good, does not support trace
%%    operator.)
%%   \item Semantic trace operator obtained via unguarded iteration in
%%     the sense of Elgot.
  \end{itemize}
  
\end{frame}

%\setbeamertemplate{section in toc}[ball unnumbered]
\begin{frame}[c]
  \begin{center}
    \Huge \Pio\ syntax
  \end{center}
\end{frame}


%\section{\Pio\ syntax}
\begin{frame}
  
  \frametitle{\Pio\ syntax: types}
  \begin{itemize}
    \item We use de Bruijn indexes to deal with type variables.
  \end{itemize}
  \[
  \arraycolsep=6pt\def\arraystretch{2.8}
  \begin{array}{cc}
    %\hline 
    \infer{\Z : \Ty\,n}{} 
    & \infer{A \oplus B : \Ty\,n}{A : \Ty\,n & B : \Ty\,n}
    \\
    \infer{\I : \Ty\,n}{}
    & \infer{A \otimes B : \Ty\,n}{A : \Ty\,n & B : \Ty\,n}
    \\
    \infer{\Var\,i : \Ty\,n}{i : \mathsf{Fin}\,n}
    & \infer{\mu A : \Ty\,n}{A : \Ty\,(n+1)}
    %\\ \hline
  \end{array}
  \]
  
\end{frame}

\begin{frame}

  \frametitle{\Pio\ syntax: programs}
  \begin{itemize}
  \item Programs of $\Pi$ [Carette and Sabry'16]:

  \end{itemize}
  \small
  \[
  \arraycolsep=-10pt\def\arraystretch{1.6}
  \begin{array}{lr}%{|cc|}
    %\hline 
    \id : A \lr A
    & \infer{g \circ f : A \lr C}{g : B \lr C & f : A \lr B} \\ [.25cm]
    \infer{f \oplus g : A \oplus B \lr C \oplus D}{f : A \lr C & g : B \lr D} 
    & \infer{f \otimes g : A \otimes B \lr C \otimes D}{f : A \lr C & g : B \lr D} \\
    \lambda_{\oplus} : \Z \oplus A \lr A : \lambda_{\oplus}^{-1}
    & \lambda_{\otimes} : \I \otimes A \lr A : \lambda_{\otimes}^{-1} \\
    %\multicolumn{2}{c}{
      \alpha_{\oplus} : (A \oplus B) \oplus C \lr A \oplus (B \oplus C) : \alpha_{\oplus}^{-1} \\
      %\multicolumn{2}{c}{
      \alpha_{\otimes} : (A \otimes B) \otimes C \lr A \otimes (B \otimes C) : \alpha_{\otimes}^{-1} \\
    \sigma_{\oplus} : A \oplus B \lr B \oplus A
    & \sigma_{\otimes} : A \otimes B \lr B \otimes A \\
    \kappa : \Z \otimes A \lr \Z : \kappa^{-1}  \\
     \delta : (A \oplus B) \otimes C \lr (A \otimes C) \oplus (B \otimes C) : \delta^{-1}
%%     \fold : A[\mu X.A/X] \lr \mu X.A
%%     & \unfold : \mu X.A \lr A[\mu X.A/X] \\[.25cm]
%%     \multicolumn{2}{c}{\infer{\trace \,f: A \lr B}{f : A \oplus C \lr B \oplus C}} \\
    %\hline
  \end{array}
  \]
\end{frame}
  
\begin{frame}

  \frametitle{\Pio\ syntax: programs (cont'd)}
  \begin{itemize}
  \item Programs of full \Pio\ also include:
    \[
    \def\arraystretch{1.6}  
    \begin{array}{c}
      \fold : A[\mu X.A/X] \lr \mu X.A : \unfold \\[.25cm]
      \infer{\trace \,f: A \lr B}{f : A \oplus C \lr B \oplus C}
      %\hline
    \end{array}
    \]
  \item The partial inverse of a programs is defined by recursion:
    \[
    \dagger : (A \lr B) \to (B \lr A)
    \]
%  \item In particular, $\dagger \,(\trace\,f) = \trace \,(\dagger \,f)$.
  \item Inversion is involutive:
    \[
    \mathsf{daggerInvol} : (f : A \lr B) \to \dagger \,(\dagger\,f) \equiv f
    \]
  \end{itemize}
\end{frame}

\begin{frame}
  \frametitle{\Pio\ syntax: program equivalences}
  \begin{itemize}    
  \item Big list of Laplaza's coherence laws of symmetric rig
    categories (already in $\Pi$ [Carette and Sabry'16]).
  \item We add program equivalences stating that:
    \begin{itemize}
    \item $(\Z,\oplus)$ is a traced monoidal structure;
    \item $\unfold$ is the total inverse of $\fold$;
    \item $\trace\,(\dagger \,t)$ is the partial inverse of $\trace\,t$;
    \item uniqueness of the partial inverses.
    \end{itemize}

    \vspace{\fill}
  \item (Altogether, the program equivalences correpond to the axioms of
    dagger-traced $\omega$-continuous rig inverse categories [Karvonen'19].)
    \end{itemize}
\end{frame}

\begin{frame}
  \frametitle{\Pio\ syntax: partial inverses}
  \begin{itemize}
  \item Here we mean ``partial inverse'' in the sense of inverse
    categories [Kastl'79].
  \item Intuitively, the partial inverse of $f$ the unique map that
    undoes everything which $f$ does.
  \item We can prove by inducton that a term $f$ has $\dagger\,f$ as
    its partial inverse:
    \[
    \mathsf{existsPIso} : (f : A \lr B) \to f \circ \dagger\,f
    \circ\,f \LR f
    \]
    \vspace{\fill}
  \item Note that $\dagger \,f \circ f$ is not always the identity,
    since generally $f$ is only a partially defined computation.
  \end{itemize}
\end{frame}

\begin{frame}[c]
  \begin{center}
    \Huge The delay datatype
  \end{center}
\end{frame}

\begin{frame}

  \frametitle{The delay monad [Capretta'05]}
  \begin{itemize}
  \item A type of possibly non-terminating computations.
  \item Coinductively defined:
    \[
    \infer{\now\,a : \Delay A}{a : A}
    \quad
    \infer={\later\,x : \Delay A}{x : \Delay A}
    \]
  \item The non-terminating computation:
    $\never = \later \,\never$.
  \item The delay datatype is a monad. Its Kleisli extension:
    \begin{align*}
      & \bind : (A \to \Delay B) \to \Delay A \to \Delay B \\
      & \bind \,f \, (\now\,a) = f\,a \\
      & \bind\,f\,(\later\,x) = \later\,(\bind\,f\,x)
    \end{align*}

  \end{itemize}
  
  
\end{frame}

\begin{frame}

  \frametitle{The Kleisli category of $\Delay$}
  \begin{itemize}

  \item Let $\D$ be the Kleisli category of $\Delay$:
    \begin{itemize}
    \item objects are types (of the metatheory);
    \item morphisms are functions in $A \to \Delay B$;
    \item the identity map is $\now$;
    \item composition of $f : A \to \Delay B$ and $g : B \to \Delay C$
      is given by:
      \[g \diamond f = \bind\,g \comp f\]
    \end{itemize}
\item Homsets in $\D$ are written: $A \to_\D B$   
  \end{itemize}
  
\end{frame}

\begin{frame}

  \frametitle{Intensional vs. extensional partiality}
  \begin{itemize}
  \item In $\D$ we can program with possibly non-terminating
    programs, but partiality is intensional.
  \item Computations terminating with the same value but 
    different number of steps are different.
  \item We quotient out finite computation steps using weak
    bisimilarity [Danielsson'18]:
    %\red{perhaps also introduce $\dn$, used in the orbit}
    \[
      \arraycolsep=10pt\def\arraystretch{2.8}
      \begin{array}{cc}
        \infer{\now\,a \approx \now\,a}{}
        &
        \infer={\later\,x_1 \approx \later\,x_2}{x_1 \approx x_2}
        \\
        \infer{\later\,x \approx \now\,a}{x \approx \now \,a}
        &
        \infer{\now\,a \approx \later\,x}{\now\,a \approx x}
      \end{array}
    \]
    \item Convergence: $x \dn a = x \approx \now\,a$.
  \end{itemize}
  
  
\end{frame}

\begin{frame}

  \frametitle{A category of partial functions}
  \begin{itemize}
    \item Let $\Dapprox$ be the category $\D$ with homsets
      quotiented by pointwise weak bisimilarity.
    \item In $\Dapprox$, two equivalence classes of morphisms $[f]$ and $[g]$ are equal whenever
      $f \, a \approx g \, a$, for all inputs $a$.
    \item Homsets in $\Dapprox$ are written: $A \to_{\Dapprox} B$   
      %    \item We write $f \approx g$ also for pointwise weak bisimilarity.
      \pause
      \vspace{\fill}
    \item We are quotienting the homsets of $\D$, not the delay datatype!
    \item So we avoid problems with semi-classical principles, such as countable choice [Chapman et al.'15].
    \item Alternatively, we could have used the partiality monad in
      HoTT [Altenkirch et al.'17].

  \end{itemize}
  
  
\end{frame}

\begin{frame}

  \frametitle{The structure of $\Dapprox$}
  \begin{itemize}
  \item Colimits in $\Dapprox$ are inherited from $\Set$.
  \item But limits in $\Set$ do not lift to limits in $\Dapprox$.
    \pause
    \vspace{\fill}
  \item Nevertheless, 1 and $\times$ are partial terminal object and
    partial binary products respectively, in the sense of Cockett and
    Lack's restriction categories.
  \item In particular, $(\Dapprox,1,\times,0,+)$ is a symmetric rig
    category.
    
  \end{itemize}
  
  
\end{frame}

\begin{frame}

  \frametitle{Partial isomorphisms}
  
  \begin{itemize}
  \item We need to restrict to reversible computations in $\Dapprox$.
  \item A morphism $f : A \to_\D B$ is a partial isomorphism if
    the following type is inhabited: %(\red{Kleene equality})
    \[
    \begin{array}{l}
      \piso \,f = \\
      \qquad (g : B \to_\D A) \times \\
      \qquad \quad \left( (a : A) (b : B) \to  f
      \,a \dn b \leftrightarrow g \,b \dn a  \right)  
    \end{array}
    \]
    \item Our definition of partial iso is equivalent to the one in
      inverse categories [Kastl'79, Cockett and Lack'02].
    \item The type of partial isos:
      \[
      A \simeq_{\mathsf{p}} B = (f : A \to_\D B) \times \piso \,f
      \]
    \item We denote $\daggerD\,f$ the partial inverse of a partial iso $f$.
  \end{itemize}
  
  
\end{frame}

\begin{frame}

  \frametitle{A category of partial isos}
  \begin{itemize}
  \item Let $\Inv \Dapprox$ be the subcategory of $\Dapprox$
    consisting of (equivalence classes of) partial isomorphisms.
  \item We write $A \simeq_{\mathsf{p}\approx} B$ for the homsets of
    $\Inv \Dapprox$.
  \item $\Inv\Dapprox$ inherits neither partial products nor
    coproducts of $\Dapprox$.
  \item However, $(\Dapprox,1,\times,0,+)$ is a symmetric rig inverse category.
  \item That is to say that $\Pi$, the finitary fragment of
    \Pio, can be modelled in $\Dapprox$.
  \end{itemize}
  
  
\end{frame}

\begin{frame}[c]
  \begin{center}
    \Huge Dagger trace in $\Inv\Dapprox$ 
  \end{center}
\end{frame}

\begin{frame}

  \frametitle{Trace via unguarded iteration}

  \begin{itemize}

  \item Following [Hasegawa'97], we can derive a trace operator from
    an unguarded iteration operator.
  \item In a category $\C$ with finite coproducts,  iteration is an operation:
    \[
    \iter : \C (A ,B + A) \to \C(A , B)
    \]
    satisfying some axioms, in particular the fixpoint equality:
    \[
    \iter\,f = \copair {\id} {\iter \,f} \circ f
    \]
  \item Pictorially:
      \ctikzfig{iter}    
  \end{itemize}
  
\end{frame}

\begin{frame}
  \frametitle{The iterator in $\Dapprox$}
  \begin{itemize}
    \item First define:
    \[ %\arraycolsep=8pt
    \def\arraystretch{1.2}
    \begin{array}{ll}
      \multicolumn{2}{l}{\iterD' : (A \to \Delay (B + A)) \to \Delay (B + A) \to \Delay B} \\
      \iterD' \,f \,(\now \,(\inl\, b)) &= \now \,b \\
      \iterD' \,f \,(\now \,(\inr\, a)) &= \later \,(\iterD' \,f \,(f \,a)) \\
      \iterD' \,f \,(\later \,x) &= \later \,(\iterD' \,f \,x) \\
    \end{array}
    \]
  \item Then take:
    \[ %\arraycolsep=8pt
    \def\arraystretch{1.2}
    \begin{array}{ll}    
      \multicolumn{2}{l}{\iterD : (A \to_\D B + A) \to (A \to_\D B)} \\
      \multicolumn{2}{l}{\iterD \,f \,a = \iterD' \,f \,(f \,a)}
    \end{array}
    \]
  \item $\iterD$ respects weak bisimilarity
    and satisfies the iteration axioms up to ${\approx}$. It lifts to
    an iterator in $\Dapprox$.
    
  \end{itemize}
  
\end{frame}

\begin{frame}
  \frametitle{The trace in $\Dapprox$}
  \begin{itemize}
    \item %% From an iterator:
      %%\ctikzfig{iter}
      We get a trace:
      \ctikzfig{trace}
      where $f = \copair {f_A}{f_C}$.
    \item Formally:
      \begin{align*}
        & \traceD : (A + C \to_\D B + C) \to (A \to_\D B) \\
        & \traceD \,f = \copair{\now}{\iterD \,f_C}\diamond f_A
      \end{align*}
%      \vspace{\fill}
  \item $\traceD$ respects weak bisimilarity and satisfies the trace
    axioms up to ${\approx}$. It lifts to an trace in $\Dapprox$.
      %Construction equivalent to the one in [Hasegawa'97].      
      %This construction is impossible in the maybe monad
  \end{itemize}
\end{frame}

\begin{frame}
  \frametitle{Dagger trace in $\Inv\Dapprox$}
  \begin{itemize}
  \item It remains to show that the trace of $\Dapprox$ restricts to
    $\Inv\Dapprox$.
  \item If $f : A + C \to_\D B + C$ is a partial isomorphism, then the
    $\traceD\,f$ is also a partial isomorphism with
    $\traceD\,(\daggerD\,f)$ as partial inverse.
  \item In the end, we want an operation:
    \[
    \semTrace : A + C\simeq_{\mathsf{p}\approx} B + C \to A \simeq_{\mathsf{p}\approx} B
    \]
    modelling the trace of \Pio.
   
  \end{itemize}
\end{frame}

\begin{frame}
  \frametitle{Dagger trace via orbit}
  \begin{itemize}
  \item The orbit of $x : A + C$ wrt. $f : A + C \to_{\D} B + C$
    is the following inductively defined type family:
    \[
    \infer{\done\,p : \Orb f x y}{p : f \,x \dn y} \quad
    \infer{\mynext\,p \,q : \Orb f x y}{p : f \,x \dn (\inr\,c) & q :
      \Orb f {(\inr\,c)} y}
    \]
  \item Therefore: %(\red{introduce $\traceD$ before})
    \[
    \traceD\,f\,a \dn b \leftrightarrow \Orb f{(\inl\,a)} {(\inl\,b)}
    \]
  \item Proof of dagger trace reduces to: 
    \[
    \Orb f {(\inl\,a)} {(\inl\,b)} \leftrightarrow \Orb
         {(\daggerD\,f)} {(\inl\,b)} {(\inl\,a)}
    \]
    \vspace{\fill}
  \item Strong resemblance between reversing an orbit and
    reversing a vector.
  \end{itemize}
\end{frame}

\begin{frame}
  \frametitle{Summing up: the interpretation}
  \begin{itemize}
  \item Types of \Pio\ are interpreted as objects of $\Inv\Dapprox$, i.e. types
    of the metatheory:
    \[
    \semTy A : (\mathsf{Fin}\,n \to \Set) \to \Set
    \]
  \item Terms of \Pio\ are modelled as morphism of $\Inv\Dapprox$,
i.e. partial isomorphisms:
    \[
    \semTm {t : A \lr B} : (\rho : \mathsf{Fin}\,n \to \Set) \to \semTy A \rho \simeq_{\mathsf{p}\approx} \semTy B \rho
    \]
  \item Term equivalences of \Pio\ are modelled as morphism equalities
    in $\Inv\Dapprox$.
%%     i.e. proofs of weak bisimilarity between two
%%     morphisms:
%%     \[
%%     \semEq {e : f \LR g} : \semTm f \approx \semTm g
%%     \]
  \end{itemize}
  
\end{frame}

\begin{frame}
  \frametitle{Conclusions}
  \begin{itemize}
  \item We have extended the work of [Carette and Sabry'16] to a
    two-level calculus of \Pio\ programs and program equivalences.
    \item This required a change of semantic domain, taking into
      account possibly non-terminating reversible computations.
    \item Alternative type-theoretical models can be constructed using
      the partiality monad [Altenkirch et al.'17] or the partial map
      classifier [Escardo and Knapp'17].
    \item Conjecture: a model for each $\omega$-continuous
      monad on $\Set$, that can hopefully be ported to other categories [Goncharov et.al'16]. 
    \item Future work: The (unquotiented) delay monad can also be used for
      semantics of time-reversible computations.
      
      
  \end{itemize}
\end{frame}

%% \begin{frame}
%%   
%%   \frametitle{Denotational semantics of \Pio}
%% 
%%   \begin{itemize}
%% 
%%   \item Categorical semantics:
%% 
%%     \begin{itemize}
%%     \item In join inverse categories [Kaarsgaard et al. 2016]
%%     \item In dagger-traced $\omega$-continuous rig categories [Bowman et al. 2011, Karvonen 2019]
%%     \end{itemize}
%% 
%%   \item Type-theoretical semantics:
%%     \begin{itemize}
%%       \item Finitary fragment of 
%%     \end{itemize}
%%   \end{itemize}
%%   
%% \end{frame}



\end{document}
